\documentclass[12pt]{article}

\usepackage[margin=1in]{geometry}

% Math and Algos
\usepackage{amsmath, amsthm, amsfonts, amssymb}
\usepackage{mathtools}
\usepackage{algorithmicx, algpseudocode}
\usepackage{verbatim}
\newcommand\R{\mathbb{R}}
\newcommand\N{\mathbb{N}}
\newcommand\Q{\mathbb{Q}}
\newcommand\Z{\mathbb{Z}}
\newcommand{\sgn}{\operatorname{sgn}}
\newtheorem{theorem}{Theorem}[section]
\theoremstyle{remark}
\newtheorem*{remark}{Remark}

% Graphics
\usepackage{graphicx}
\usepackage{color}
\graphicspath{{./images/}}
\usepackage{booktabs}

% Hyperlinks
\usepackage{hyperref}
\hypersetup{
    linkcolor=cyan
}

% Aesthetics
\usepackage{enumitem}
\allowdisplaybreaks
\hfuzz=14pt

\begin{document}

\title{M 361K Homework 3}
\author{Ishan Shah}
\date{\today}
\maketitle

%%%%%%%%%% 5.1 %%%%%%%%%%
\section*{5.1}
\paragraph{3.} Let $a<b<c$. Suppose that $f$ is continuous on $[a, b]$, that $g$ is continuous on $[b, c]$, and that $f(b)=g(b)$. Define $h$ on $[a, c]$ by $h(x):=f(x)$ for $x \in[a, b]$ and $h(x):=g(x)$ for $x \in[b, c]$. Prove that $h$ is continuous on $[a, c]$.
\begin{proof}
    We have that $h$ is continuous from $[a, b) \cup (b, c]$ from the definition of $h$ and the continuity of $f$ and $g$. We now need to show that $h$ is continuous at $b$. Now, $\lim_{x \to b^-} = \lim_{x \to b^-} = f(b)$ and $\lim_{x \to b^+} = \lim_{x \to b^+} = g(b)$. 
    
    Because $f(b) = g(b)$, $\lim_{x \to b^-} h(x) = \lim_{x \to b^+} h(x) = h(b)$. Thus, $\lim_{x \to b} h(x) = h(b)$. Therefore, since $h$ is continuous from $[a, b) \cup [b] \cup (b, c]$, $h$ is continuous on $[a, c]$.
\end{proof}

\paragraph{7.} Let $f: \mathbb{R} \rightarrow \mathbb{R}$ be continuous at $c$ and let $f(c)>0$. Show that there exists a neighborhood $V_\delta(c)$ of $c$ such that if $x \in V_\delta(c)$, then $f(x)>0$.
\begin{proof}
    Let $\epsilon = \frac{f(c)}{2} > 0$. Then, let there be some $\delta > 0$ such that there exists some $V_\delta(c) = (c - \delta, c + \delta)$. Then, let $x \in V_\delta(c) = (c - \delta, c + \delta)$. Then,
    \begin{align*}
        x \in (c - \delta, c + \delta) &\implies |f(x) - f(c)| < \epsilon \\
        &\implies -\epsilon < f(x) - f(c) < \epsilon \\
        &\implies f(c) - \frac{f(c)}{2} < f(x) \\
        &\implies f(x) > 0
    \end{align*}
    Thus, $f(x) > 0$ for all $x \in V_\delta(c)$.
\end{proof}

\paragraph{12.} Suppose that $f: \mathbb{R} \rightarrow \mathbb{R}$ is continuous on $\mathbb{R}$ and that $f(r)=0$ for every rational number $r$. Prove that $f(x)=0$ for all $x \in \mathbb{R}$.
\begin{proof}
    Since $\Q$ is dense in $\R$, we can find some sequence of rational numbers $(x_n)$ that converges to some $x$ in $\R$. We know that $f$ is continuous at $x$, so we can say that $(f(x_n))$ converges to $f(x)$ and $f(x_n) = 0 \;\forall\; n \in \N$ because $f(r) = 0 \;\forall\; r \in \Q$. Thus, $f(x) = \lim f(x_n) = \lim(0) = 0$.
\end{proof}

%%%%%%%%%% 5.2 %%%%%%%%%%
\section*{5.2}
\paragraph{2.} Show that if $f: A \rightarrow \mathbb{R}$ is continuous on $A \subseteq \mathbb{R}$ and if $n \in \mathbb{N}$, then the function $f^n$ defined by $f^n(x)=(f(x))^n$, for $x \in A$, is continuous on $A$.
\begin{proof}
    We can do a proof by induction where the base case is $n = 1$. Then, $f^1(x) = f(x)$ which is trivially continuous on $A$ by our original assumption.
    
    Our inductive case is that $f^{n + 1} = f^n f$ which is true by Theorem 5.2.1 which states that the product of two continuous functions is continuous. Thus, $f^{n + 1}$ is continuous on $A$.
\end{proof}

\paragraph{3.} Give an example of functions $f$ and $g$ that are both discontinuous at a point $c$ in $\mathbb{R}$ such that the sum $f + g$ is continuous at $c$ and the product $f \cdot g$ is continuous at $c$.
\begin{proof}
    Let $f(x)$ and $g(x)$ be defined as follows:
    $$f(x) = \begin{cases}
        1 & \text{if } x = c \\
        0 & \text{if } x \neq c
    \end{cases}$$
    $$g(x) = \begin{cases}
        0 & \text{if } x = c \\
        1 & \text{if } x \neq c
    \end{cases}$$
    Then, $f + g$ is continuous at $c$ because $f(c) + g(c) = 1$ and $f \cdot g$ is continuous at $c$ because $f(c) \cdot g(c) = 0$ for all values of $x$.
\end{proof}

\paragraph{7.} Give an example of a function $f:[0,1] \rightarrow \mathbb{R}$ that is discontinuous at every point of $[0,1]$ but such that $|f|$ is continuous on $[0,1]$.
\begin{proof}
    The Density Theorem tells us that for every pair of rational numbers, there exists an irrational number between them, and vice versa. Thus, let $f$ be defined as follows:
    $$f(x) = \begin{cases}
        -1 & \text{if } x = \frac{p}{q} \text{ for some } p, q \in \mathbb{N} \\
        \phantom{-}1 & \text{if } x \neq \frac{p}{q} \text{ for all } p, q \in \mathbb{N}
    \end{cases}$$
    Then, $|f|$ is 1 everywhere. Thus, $f$ is discontinuous at every point of $[0,1]$ but $|f|$ is continuous on $[0,1]$.
\end{proof}

\paragraph{8.} Let $f, g$ be continuous from $\mathbb{R}$ to $\mathbb{R}$, and suppose that $f(r)=g(r)$ for all rational numbers $r$. Is it true that $f(x)=g(x)$ for all $x \in \mathbb{R}$?
\begin{proof}
    The Density Theorem states that there is a irrational number between every pair of rational numbers. Because $f(r) = g(r)$ for the rationals, $f(s) = g(s)$ for all irrational numbers $s$ in order to maintain continuity for $f$ and $g$ using the Density Theorem. Thus, $f(x) = g(x)$ for all $x \in \mathbb{R}$ is true.
\end{proof}

%%%%%%%%%% 6.1 %%%%%%%%%%
\section*{6.1}
\paragraph{3.} Let $I \subseteq \mathbb{R}$ be an interval, let $c \in I$, and let $f: I \rightarrow \mathbb{R}$ and $g: I \rightarrow \mathbb{R}$ be functions that are differentiable at $c$. Prove the following:
\begin{enumerate}[label=(\alph*)]
    \item If $\alpha \in \mathbb{R}$, then the function $\alpha f$ is differentiable at $c$.
    \begin{proof}
        We can use the definition of the derivative to show this.
        \begin{align*}
            (\alpha f)'(c) &= \lim_{x \to c} \frac{\alpha f(x) - \alpha f(c)}{x - c} \\
            &= \alpha \lim_{x \to c} \frac{f(x) - f(c)}{x - c} \\
            &= \alpha f'(c)
        \end{align*}
    \end{proof}

    \item The function $f + g$ is differentiable at $c$ and $(f + g)^{\prime}(c) = f^{\prime}(c) + g^{\prime}(c)$.
    \begin{proof}
        \begin{align*}
            (f + g)'(c) &= \lim_{x \to c} \frac{(f + g)(x) - (f + g)(c)}{x - c} \\
            &= \lim_{x \to c} \frac{f(x) + g(x) - f(c) - g(c)}{x - c} \\
            &= \lim_{x \to c} \frac{f(x) - f(c)}{x - c} + \lim_{x \to c} \frac{g(x) - g(c)}{x - c} \\
            &= f'(c) + g'(c)
        \end{align*}
    \end{proof}
\end{enumerate}

\paragraph{4.} Let $f: \mathbb{R} \rightarrow \mathbb{R}$ be defined by $f(x):=x^2$ for $x$ rational, $f(x):=0$ for $x$ irrational. Show that $f$ is differentiable at $x=0$, and find $f^{\prime}(0)$.
\begin{proof}
    Let $g(x) = \frac{f(x) - f(0)}{x - 0} = \frac{f(x)}{x}$ so $g(x) := x$ for $x$ rational and $g(x) := 0$ for $x$ irrational. To show that $f$ is differentiable at 0 we have to show that $\lim_{x \to 0} g(x) = f'(0)$ exists. 
    
    We know that $-|x| \leq g(x) \leq |x|$ for all $x \in \mathbb{R}$ and that $\lim_{x \to 0} -|x| = -0$ and $\lim_{x \to 0} |x| = 0$. Thus, $\lim_{x \to 0} g(x) = 0$ by Theorem 3.2.7 (Squeeze Theorem) so $f$ is differentiable at $x = 0$ and $f'(0) = 0$.
\end{proof}
\pagebreak

\paragraph{7.} Suppose that $f: \mathbb{R} \rightarrow \mathbb{R}$ is differentiable at $c$ and that $f(c)=0$. Show that $g(x):=|f(x)|$ is differentiable at $c$ if and only if $f^{\prime}(c)=0$.
\begin{proof}
    We want to show that if $g(x)$ is differentiable at $c$, then $f'(c) = 0$. Suppose not. Suppose that $g(x)$ is differentiable at $c$ and $f'(c) \neq 0$. This means that the function $f(c)$ in some neighborhood around $c$ must cross the $x$-axis at least once in order to satisfy the conditions that $f(c) = 0$ and $f'(c) \neq 0$ using the Intermediate Value Theorem. When we take the absolute value of $f(x)$ to produce $g(x)$ we must reflect the negative portion of $f(x)$ over the $x$-axis, creating a cusp at $c$. This means $g(x)$ is not differentiable at $c$, which is a contradiction. Thus, $f'(c) = 0$.

    We want to show that if $f'(c) = 0$, then $g(x)$ is differentiable at $c$. For a function to be differentiable at a point, the function must be continuous and the slope of the tangent line at the point must equal the limit of the function as $x$ approaches the point. Because $f'(c) = 0$, the slope of the tangent line at $c$ is 0. Because $g(x)$ is continuous at $c$, the limit of $g(x)$ as $x$ approaches $c$ must also be 0. Thus, $g(x)$ is differentiable at $c$.

    Using both parts of this proof we can show that $g(x)$ is differentiable at $c$ if and only if $f'(c) = 0$.
\end{proof}

\paragraph{9.} Prove the following:
\begin{enumerate}[label=(\alph*)]
    \item If $f: \mathbb{R} \rightarrow \mathbb{R}$ is a differentiable even function, then the derivative $f^{\prime}$ is an odd function.
    \begin{proof}
        Because $f$ is even, $f(x) = f(-x) \;\forall\; x \in \mathbb{R}$.
        \begin{align*}
            f'(-c) &= \lim_{x \to -c} \frac{f(x) - f(-c)}{x - (-c)} \\
            &= \lim_{x \to c} \frac{f(-x) - f(c)}{-x + c} \\
            &= \lim_{x \to c} \frac{f(x) - f(c)}{-(x - c)} \\
            &= -\lim_{x \to c} \frac{f(x) - f(c)}{x - c} \\
            &= -f'(c)
        \end{align*}
        Then, $f'(-c) = -f'(c) \;\forall\; c \in \mathbb{R}$, so $f'$ is odd.
    \end{proof}
    
    \item If $g: \mathbb{R} \rightarrow \mathbb{R}$ is a differentiable odd function, then the derivative $g^{\prime}$ is an even function.
    \begin{proof}
        Because $g$ is odd, $g(-x) = -g(x) \;\forall\; x \in \mathbb{R}$.
        \begin{align*}
            g'(-c) &= \lim_{x \to -c} \frac{g(x) - g(-c)}{x - (-c)} \\
            &= \lim_{x \to c} \frac{g(-x) - (-g(c))}{-x + c} \\
            &= \lim_{x \to c} \frac{-g(x) + g(c)}{-x + c} \\
            &= \lim_{x \to c} \frac{-(g(x) - g(c))}{-(x - c)} \\
            &= \lim_{x \to c} \frac{g(x) - g(c)}{x - c} \\
            &= g'(c)
        \end{align*}
        Then, $g'(-c) = g'(c) \;\forall\; c \in \mathbb{R}$, so $g'$ is even.
    \end{proof}
\end{enumerate}

%%%%%%%%%% 6.2 %%%%%%%%%%
\section*{6.2}
\paragraph{5.} Let $a>b>0$ and let $n \in \mathbb{N}$ satisfy $n \geq 2$. Prove that $a^{1 / n}-b^{1 / n}<(a-b)^{1 / n}$. \emph{Hint: Show that $f(x):=x^{1 / n}-(x-1)^{1 / n}$ is decreasing for $x \geq 1$, and evaluate $f$ at 1 and $a / b$.}
\begin{proof}
    Let $f(x):=x^{1 / n}-(x-1)^{1 / n}$. Then, $f^{\prime}(x)=\frac{1}{n}(x^{\frac{1}{n}-1}-(x-1)^{\frac{1}{n}-1})$. Since $n \geq 2$, $1-\frac{1}{n} \geq \frac{1}{2}$ which means that $x^{1-\frac{1}{n}}>(x-1)^{1-\frac{1}{n}}$ when $x>1$. Thus, $f^{\prime}(x)<0$ when $x>1$, so $f(x)$ is decreasing when $x \geq 1$. For $a>b>0$, we get $\frac{a}{b}>1$ so that $f(\frac{a}{b})=(\frac{a}{b})^{1 / n}-(\frac{a}{b}-1)^{1 / n}<f(1)=1$, i.e. $a^{1 / n}-b^{1 / n}<(a-b)^{1 / n}$.
\end{proof}

\paragraph{6.} Use the Mean Value Theorem to prove that $|\sin x-\sin y| \leq |x-y|$ for all $x, y$ in $\mathbb{R}$.
\begin{proof}
    The Mean Value Theorem states $f'(c) = \frac{f(b) - f(a)}{b - a} \implies f(b) - f(a) = f'(c)(b - a)$. We can apply the $\sin$ function here because it is continuous and differentiable. Then,
    \begin{align*}
        |\sin x - \sin y| &= |\cos c (x - y)| \\
        \frac{|\sin x - \sin y|}{|x - y|} &= |\cos c| \\
    \end{align*}
    Since $\cos c$ is bounded by 1, we get that $$\frac{|\sin x - \sin y|}{|x - y|} \leq 1$$ Therefore, $|\sin x - \sin y| \leq |x - y|$.
\end{proof}

\paragraph{8.} Let $f:[a, b] \rightarrow \mathbb{R}$ be continuous on $[a, b]$ and differentiable in $(a, b)$. Show that if $\lim _{x \rightarrow a} f^{\prime}(x)=A$, then $f^{\prime}(a)$ exists and equals $A$. \emph{Hint: Use the definition of $f^{\prime}(a)$ and the Mean Value Theorem.}
\begin{proof}
    $f^{\prime}(a) = \lim_{x \to a} \frac{f(x) - f(a)}{x - a}$ by the definition of the derivative. Using the Mean Value Theorem, we have that $f'(c) = \frac{f(x) - f(a)}{x - a}$. Then, $$\lim_{x \to a} \frac{f(x) - f(a)}{x - a} = \lim_{c \to a} f'(x) = A$$
    Thus, $f^{\prime}(a)$ exists and equals $A$.
\end{proof}

\paragraph{10.} Let $g: \mathbb{R} \rightarrow \mathbb{R}$ be defined by $g(x):=x+2 x^2 \sin (1 / x)$ for $x \neq 0$ and $g(0):=0$. Show that $g^{\prime}(0)=1$, but in every neighborhood of 0 the derivative $g^{\prime}(x)$ takes on both positive and negative values. Thus $g$ is not monotonic in any neighborhood of 0.
\begin{proof}
    The derivative $g'(x) = 4x\sin(1/x) - 2\cos(1/x) + 1$. We can compute $g'(0)$ from the limit definition:
    \begin{align*}
        g'(0) &= \lim_{x \to 0} \frac{g(x) - g(0)}{x - 0} \\
        &= \lim_{x \to 0} \frac{x + 2x^2\sin(1/x) - 0}{x - 0} \\
        &= \lim_{x \to 0} 1 + 2x\sin(1/x) \\
        &= 1
    \end{align*}
    Then, we can let $x_n = \frac{1}{2n\pi}$ and $y_n = \frac{1}{(4n + 1)\pi}$. This means $g'(x_n) = -1 < 0$ and $g'(y_n) ~ 1 + \frac{4}{(4n + 1) \frac{\pi}{2}} > 0$. Thus, in every neighborhood of 0 $g'(x)$ takes both positive and negative values so $g$ is not monotonic in any neighborhood of 0.
\end{proof}

\paragraph{13.} Let $I$ be an interval and let $f: I \rightarrow \mathbb{R}$ be differentiable on $I$. Show that if $f^{\prime}$ is positive on $I$, then $f$ is strictly increasing on $I$.
\begin{proof}
    Let there be some $x, y \in I$ such that $x < y$. From the Mean Value Theorem we have that $f(y) - f(x) = f'(c)(y - x)$. Since $f'$ is positive on $I$, $f'(c)(y - x) > 0$. Thus, $f(y) - f(x) > 0$ for all $x, y \in I$ so $f$ is strictly increasing on $I$.
\end{proof}

\end{document}