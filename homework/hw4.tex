\documentclass[12pt]{article}

\usepackage[margin=1in]{geometry}

% Math and Algos
\usepackage{amsmath, amsthm, amsfonts, amssymb}
\usepackage{mathtools}
\usepackage{algorithmicx, algpseudocode}
\usepackage{verbatim}
\newcommand\R{\mathbb{R}}
\newcommand\N{\mathbb{N}}
\newcommand\Q{\mathbb{Q}}
\newcommand\Z{\mathbb{Z}}
\newcommand{\sgn}{\operatorname{sgn}}
\newtheorem{theorem}{Theorem}[section]
\theoremstyle{remark}
\newtheorem*{remark}{Remark}

% Graphics
\usepackage{graphicx}
\usepackage{color}
\graphicspath{{./images/}}
\usepackage{booktabs}

% Hyperlinks
\usepackage{hyperref}
\hypersetup{
    linkcolor=cyan
}

% Aesthetics
\usepackage{enumitem}
\allowdisplaybreaks
\hfuzz=14pt

\begin{document}

\title{M 361K Homework 4}
\author{Ishan Shah}
\date{\today}
\maketitle

%%%%%%%%%% 6.3 %%%%%%%%%%
\section*{6.3}
\paragraph{13.} Try to use L'Hospital's Rule to find the limit of $\frac{\tan x}{\sec x}$ as $x \rightarrow(\pi / 2)-$. Then evaluate directly by changing to sines and cosines.

\paragraph{14.} Show that if $c>0$, then $\lim _{x \rightarrow c} \frac{x^c-c^x}{x^x-c^c}=\frac{1-\ln c}{1+\ln c}$.

%%%%%%%%%% 6.4 %%%%%%%%%%
\section*{6.4}
\paragraph{11.} If $x \in[0,1]$ and $n \in \mathbb{N}$, show that
$$
\left|\ln (1+x)-\left(x-\frac{x^2}{2}+\frac{x^3}{3}+\cdots+(-1)^{n-1} \frac{x^n}{n}\right)\right|<\frac{x^{n+1}}{n+1} .
$$
Use this to approximate $\ln 1.5$ with an error less than $0.01$. Less than $0.001$.

\paragraph{13.} Calculate $e$ correct to 7 decimal places.

%%%%%%%%%% 7.1 %%%%%%%%%%
\section*{7.1}
\paragraph{2.} If $f(x):=x^2$ for $x \in[0,4]$, calculate the following Riemann sums, where $\dot{\mathcal{P}}_i$ has the same partition points as in Exercise 1, and the tags are selected as indicated. $\mathcal{P}_2 := (0, 2, 3, 4)$.
\begin{itemize}
    \item $\dot{\mathcal{P}}_2$ with the tags at the left endpoints of the subintervals.
    \item $\dot{\mathcal{P}}_2$ with the tags at the right endpoints of the subintervals.
\end{itemize}

\paragraph{6b.} Let $h(x):=2$ if $0 \leq x<1, h(1):=3$ and $h(x):=1$ if $1<x \leq 2$. Show that $h \in \mathcal{R}[0,2]$ and evaluate its integral.

\paragraph{8.} If $f \in \mathcal{R}[a, b]$ and $|f(x)| \leq M$ for all $x \in[a, b]$, show that $\left|\int_a^b f\right| \leq M(b-a)$.

\paragraph{10.} Let $g(x):=0$ if $x \in[0,1]$ is rational and $g(x):=1 / x$ if $x \in[0,1]$ is irrational. Explain why $g \notin \mathcal{R}[0,1]$. However, show that there exists a sequence $\left(\dot{\mathcal{P}}_n\right)$ of tagged partitions of $[a, b]$ such that $\left\|\dot{\mathcal{P}}_n\right\| \rightarrow 0$ and $\lim _n S\left(g ; \dot{\mathcal{P}}_n\right)$ exists.

\end{document}