\section{September 6}

\subsection{Cardinality}
\begin{definition}{Cardinality}{}
    The cardinality of a set $A$ is the number of elements in $A$. We denote this as $|A|$. We say that two sets $A$ and $B$ have the same cardinality if and only if $\exists$ a bijection $f: A \rightarrow B$, or $|A| = |B|$.
\end{definition}
\begin{note}
    This bijection holds true because cardinality is reflexive (via the identity function), symmetric (via the inverse function), and transitive (via composition).
\end{note}

\begin{note}
    The following examples demonstrate how to prove whether two sets have the same cardinality.
\end{note}
\begin{itemize}
    \item $|\textnormal{even integers}| = |\textnormal{odd integers}|$: $f(2n) = 2n + 1$.
    \item $|\Z| = |\Z^+|$: $f(0) = 1$, $f(1) = 2$, $f(-1) = 3$, $f(2) = 4$, $\ldots$
    \item $|\Q^+| = |\Z^+|$: We can create a diagonal mapping by taking $\frac{n}{m}$ for counting numbers on the rows and columns.
    \item $|\Q| = |\Z^+|$: $\Q = \Q^+ \cup \Q^- \cup \{0\}$, so we can repeat the diagonal mapping for $\Q^-$. This is because any subset of a countable set is countable.
    \item $|\Q| \neq |\R|$: For the real numbers, Cantor's Diagonal Argument proves the sets have different cardinality since no possible surjection exists.
\end{itemize}
In essence, if we show that there exists some one-to-one mapping between the two sets we can claim that $|A| = |B|$.

\subsection{Countability}
\begin{definition}{Countable}{}
    If a set is finite or has the same cardinality as $\N$ (i.e. $\Z^+)$, we say that the set is countable.
\end{definition}

\begin{theorem}{}{}
    Any subset of a countable set is countable.
\end{theorem}
\begin{theorem}{}{}
    Any set that contains an uncountable set is uncountable.
\end{theorem}

\begin{theorem}{}{}
    If $[a_n, b_n] \forall n \in \N$ is a nested sequence of closed bounded intervals, $\exists \delta \in \R$ such that $\delta \in I_n \forall n \in \N$.
\end{theorem}
\begin{proof}
    $I_n \subseteq I_1 \forall n \in \N$. Thus, $a_n \subseteq b_1 \forall n \in \N$. So, $b_n$ is an upper bound for $\{a_n \mid n \in \N\}$. Let $\delta$ be the supremum of $\{a_n \mid n \in \N\}$. Thus, $a_n \leq \delta \forall n \in \N$.

    We have now shown that $a_n \leq \delta \forall n \in \N$, and we need to show that $\delta \leq b_n \forall n \in \N$. This is left as an exercise for the reader.
\end{proof}
\begin{note}
    A nested sequence means that successive subsets contain the previous subset. For example, $[0, 1] \subseteq [0, 2] \subseteq [0, 3] \subseteq \ldots$ is a nested sequence.
\end{note}

\begin{theorem}{}{}
    $[0, 1]$ is uncountable.
\end{theorem}
\begin{proof}
    Assume $[0, 1]$ is countable. That is, $[0, 1] = I = \{x_1, x_2, x_3, \ldots\}$. Select a closed interval $I_1 \subseteq I$ such that $x_1 \not\in I_1$. Next, select a closed interval $I_2 \subseteq I_1$ such that $x_2 \not\in I_2$, and so on. Then, we have $$ I_n \subseteq \ldots \subseteq I_2 \subseteq I_1 \subseteq I $$ and $x_n \not\in I_n \forall n \in \N$. By \textbf{Theorem 3.3}, $\exists \delta \in I$ such that $\delta \in I_n \forall n \in \N$. This implies that $\delta \neq x_n \forall n \in \N$. Thus, $\delta \not\in I$, which is a contradiction. Therefore, $[0, 1]$ is uncountable.
\end{proof}