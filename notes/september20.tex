\chapter{September 20}

\section{Empty Set}
\begin{theorem}{}{}
    The empty set is a subset of any set.
\end{theorem}
\begin{proof}
    Suppose not. That is, suppose $\exists A$ such that $\emptyset \not\subset A$. Thus, $\exists x \in \emptyset$ such that $x \not\in A$. This is a contradiction because the empty set has no elements. Therefore, $\emptyset \subset A$.
\end{proof}

\begin{theorem}{}{}
    There is only one set with no elements.
\end{theorem}
\begin{proof}
    Suppose not. That is, suppose $\exists$ two empty sets $E_1, E_2$. Then $E_1 \subseteq E_2$ and $E_2 \subseteq E_1$. Thus, $E_1 = E_2$. This is a contradiction because $E_1$ and $E_2$ are two different sets. Therefore, there is only one empty set.
\end{proof}
\begin{note}{Closedness of $\emptyset$}{}
    The empty set is open and closed (vacuously true).
\end{note}

\section{Topology}
Let $S \subseteq \R$ for the following definitions.

\begin{definition}{Neighborhood}{}
    A neighborhood of $x$ in $S$ can be thought of an varepsilon-sized ball around $x$, i.e. $N(x, \varepsilon) = \{ y \in R \mid 0 \leq |x - y| < \varepsilon \}$.
\end{definition}
\begin{definition}{Deleted neighborhood}{}
    A deleted neighborhood is the same as a neighborhood except that $x$ is not included, i.e. $N^\star(x, \varepsilon) = \{ y \in R \mid 0 < |x - y| < \varepsilon \}$.
\end{definition}

\begin{definition}{Accumulation point}{}
    $x \in \R$ is an accumulation point of $S$ if and only if every deleted neighborhood of $x$ contains a point of $S$.
\end{definition}
\begin{note}
    $(0, \infty)$ has accumulation points $[0, \infty)$. $(0, 1)$ does not contain all of its accumulation points since 0 and 1 are both accumulation points of the set.
\end{note}

\begin{theorem}{}{}
    $S \in \R$ is closed if and only if $S$ contains all of its accumulation points.
\end{theorem}
\begin{proof}
    Suppose $S$ is closed. Let $x$ be an accumulation point of $S$. If $x \not\in S$, then $x \in S^\complement$. Thus, $\exists$ a neighborhood $N$ of $x$ such that $N \subseteq S^\complement$. But $N \cap S = \emptyset$, which contradicts $x$ being an accumulation point of $S$.

    Conversely, suppose $S$ contains all of its accumulation points. Let $x \in S^\complement$, then $x$ is not an accumulation point of $S$. Thus, $\exists N^\star(x, \varepsilon)$ that misses $S$. Since $x \not\in S$, $N(x, \varepsilon)$ misses $S$. Therefore, $S^\complement$ is open, which means $S$ is closed.
\end{proof}

\begin{theorem}{}{}
    If $S$ is a nonempty closed bounded subset of $\R$, then $S$ has a max.
\end{theorem}
\begin{proof}
    Let $s = \sup S$. Then, $s$ is an accumulation point of $S$. Since $S$ is closed, $s \in S$. Thus, $s$ is a max of $S$.
\end{proof}

\begin{definition}{Interior point}{}
    $x \in S$ is an interior point of $S$ if and only if $\exists N(x, t)$ such that $N(x, t) \subset S$.
\end{definition}

\begin{definition}{Boundary point}{}
    $x \in S$ is a boundary point of $S$ if and only if every neighborhood $N$ of $x$ has $N \cap S \neq \emptyset$ and $N \cap S^\complement \neq \emptyset$.
\end{definition}

\section{Closure}
\begin{definition}{Open set}{}
    $S$ is an open set if and only if every point in $S$ is an interior point of $S$. $\forall x \in S, \exists$ a neighborhood $N(x, \varepsilon)$ for some $\varepsilon > 0$ such that $N(x, \varepsilon) \subseteq S$.
\end{definition}

\begin{definition}{Closed set}{}
    $S$ is a closed set if and only $S$ contains at least one of its boundary points. Additionally, $S^\complement$ must be an open set.
\end{definition}

\begin{note}
    $\R$ is open because all of its points are interior points. $\R$ is also closed because $\R$ has no boundary points, therefore implying that it contains at least one of its boundary points (vacuously true).
\end{note}

\begin{theorem}{}{}
    The union of two open sets is open.
\end{theorem}
\begin{proof}
    Let $A$ and $B$ be open sets. Let $x \in A \cup B$. Then $x \in A$ or $x \in B$. If $x \in A$, then $\exists$ a neighborhood $N_1$ of $x$ such that $N_1 \subseteq A$. But then, $N_1 \subseteq A \cup B$. If $x \in B$, then $\exists$ a neighborhood $N_2$ of $x$ such that $N_2 \subseteq B$. But then, $N_2 \subseteq A \cup B$. 
    
    Thus, in either case, $\exists$ a neighborhood $N$ of $x$ such that $N \subseteq A \cup B$. Therefore, $A \cup B$ is open.
\end{proof}

\begin{theorem}{}{}
    An arbitrary union of open sets is open.
\end{theorem}
\begin{proof}
    Let $A_1, A_2, \ldots, A_n$ be open sets. Let $x \in \bigcup_{i=1}^n A_i$. Then $x \in A_i$ for some $i$. Let $N_i$ be a neighborhood of $x$ such that $N_i \subseteq A_i$. Then $N_i \subseteq A_i \subseteq \bigcup_{i=1}^n A_i$. Therefore, $\bigcup_{i=1}^n N_i \subseteq \bigcup_{i=1}^n A_i$. 
    
    Thus, $\bigcup_{i=1}^n N_i$ is a neighborhood of $x$ such that $\bigcup_{i=1}^n N_i \subseteq \bigcup_{i=1}^n A_i$. Therefore, $\bigcup_{i=1}^n A_i$ is open.
\end{proof}

\begin{theorem}{}{}
    The intersection of two open sets is open.
\end{theorem}
\begin{proof}
    Let $A$ and $B$ be open sets. Let $x \in A \cap B$. Then $x \in A$ and $x \in B$. Thus, $\exists$ neighborhoods $N_1(x, \varepsilon_1)$ and $N_2(x, \varepsilon_2)$. Let $\varepsilon = min\{\varepsilon_1, \varepsilon_2\}$. Then $N_1(x, \varepsilon) \subseteq A$ and $N_2(x, \varepsilon) \subseteq B$. 
    
    Thus, $N(x, \varepsilon) \subseteq A \cap B$. Therefore, $A \cap B$ is open.
\end{proof}

\begin{theorem}{}{}
    A finite intersection of open sets is open.
\end{theorem}
\begin{proof}
    Let $A_1, A_2, \ldots, A_n$ be open sets. Let $x \in \bigcap_{i=1}^n A_i$. Then $x \in A_i$ for all $i$. Let $N_i$ be a neighborhood of $x$ such that $N_i \subseteq A_i$. Then $N_i \subseteq A_i \subseteq \bigcap_{i=1}^n A_i$. Therefore, $\bigcap_{i=1}^n N_i \subseteq \bigcap_{i=1}^n A_i$. 
    
    Thus, $\bigcap_{i=1}^n N_i$ is a neighborhood of $x$ such that $\bigcap_{i=1}^n N_i \subseteq \bigcap_{i=1}^n A_i$. Therefore, $\bigcap_{i=1}^n A_i$ is open.
\end{proof}

\begin{theorem}{}{}
    An arbitrary intersection of open sets is open.
\end{theorem}
\begin{note}
    $\bigcap_{i=1}^\infty (-\frac{1}{n}, \frac{1}{n}) = \emptyset$.
\end{note}